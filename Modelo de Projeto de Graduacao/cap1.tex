\chapter{Introdução}
\label{cap1}

O presente capítulo tem como objetivo apresentar brevemente o escopo do trabalho desenvolvido assim como sua motivação, enumerando conceitos pertinentes a área de conhecimento.

\section{Tema}

\paragraph{} Este projeto tem como tema o estudo de técnicas para anonimização e dessensibilização em conjuntos de dados. Especificamente, serão avaliadas diferentes técnicas e suas implementações em software de código aberto.

Este é um trabalho majoritariamente de Engenharia de Software, contemplando um ciclo de vida para planejamento, pesquisa e implementação da solução proposta.

É possível dizer que se trata de um trabalho da área de Engenharia de Dados, dada a natureza das entidades que serão manipuladas. Também serão vistos conceitos da área de Segurança Digital, além de Direito Digital e Privacidade.

Assim, este projeto implementa uma nova ferramenta de anonimização e dessensibilização de código aberto.


\section{Delimitação}

\paragraph{} Este trabalho se limita a estudar técnicas de anonimização, garantia de privacidade e dessensibilização, criadas até o presente momento de sua concepção, que permitam compatibilizar um banco de dados relacional contendo informações sensíveis a uso por terceiros para o propósito do desenvolvimento de aplicações. As implementações destas técnicas, quando existentes, serão estudadas a partir de softwares de código aberto.

\section{Justificativa}

\paragraph{} A sociedade atual existe num momento em que a conectividade é amplamente difundida e contínua aonde um indíviduo mantém-se constantemente conectado a outros individuos e instituições através de serviços digitais.
\paragraph{} Esses serviços digitais são acessíveis via internet e acumulam um grande volume de dados sobre seus usuários. Ações indevidas habilitadas por estes dados podem trazer prejuizos para os indivíduos e para a sociedade como um todo.
\paragraph{} O risco trazido pela possibilidade de um abuso desses dados levou a criação de diversas regulações ao redor do mundo. A adaptação de instituições a essas regulações protege a sociedade mas gera uma complexidade adicional em projetos de engenharia de software, que passam a estar sob exigências mais rígidas sobre o acesso a esses dados.
\paragraph{} Assim, surge a necessidade de ferramentas que auxiliem no cumprimento dessas regulações, protejam a privacidade dos usuários e desonerem o processo de desenvolvimento de software.


\section{Objetivos}

\paragraph{} O objetivo geral deste trabalho é implementar um software livre que permita processar um conjunto de dados tornando-o dessensibilizado, sendo desta forma possível utiliza-lo para apoiar atividades de desenvolvimento e testes de sistemas sem a preocupação com vazamento de dados por parte de terceiros.
Especificamente, o software deve: (1) Remover informações que permitam associar indivíduos com um conjunto de dados; (2) Permitir a substituição de dados reais sensíveis por dados gerados a partir de estatísticas; (3)Respeitar e manter a estrutura do modelo de dados existente, alterando somente o seu conteúdo.


\section{Metodologia}

\paragraph{} TODO Como é a abordagem do assunto. Como foi feita a pesquisa, se vai houve validação, etc. Em resumo, você de explicar qual foi sua estratégia para atender ao objetivo do trabalho (tamanho do texto: livre).


\section{Descrição}

\paragraph{}No capítulo 2 será feita a apresentação introdutória sobre privacidade, conceitos associados e suas relevâncias no contexto socioeconômico contemporâneo.

\paragraph{}O capítulo 3 se dedica a apresentar a interseção do assuntos da engenharia e da privacidade de dados, trazendo um panorama das definições e técnicas existentes. Além disso, o capítulo também discute ferramentas existentes, limitadas a aquelas cuja licença garante a abertura do código-fonte, com o intuito de prospectar as funcionalidades fornecidas por esses programas e possíveis cenários de aplicação.

\paragraph{}O capítulo 4 propõe um software a ser construído e define as suas especificações, mostrando as decisões técnicas tomadas, a arquitetura e explicando seus diferentes componentes.

\paragraph{}O capítulo 5 explica como foi realizado o processo de validação do software construído e demonstra seus casos de uso.

\paragraph{}O capítulo 6 apresenta a conclusão deste trabalho.
