\documentclass[a4paper,12pt,oneside,openany]{book}
\input{TesePack}

\newcommand{\titulo}{FERRAMENTA PARA TRATAMENTO DE INFORMAÇÕES SENSÍVEIS EM BANCOS DE DADOS}

\begin{document}

\frontmatter
\thispagestyle{empty}

\includegraphics[scale=0.7]{Poli.eps}

\begin{center}
\large{\titulo{}}\\
   \vspace{2cm}
\large{Varlen Pavani Neto}\\
\end{center}
   \vspace{3cm}
\hspace{7cm}
\hfill \parbox{8.0cm}{Projeto de Graduação apresentado ao Curso de Engenharia Eletrônica e de Computação da Escola Politécnica, Universidade Federal do Rio de Janeiro, como parte dos requisitos necessários à obtenção do título de Engenheiro.\\}
   \vspace{2cm}
\hfill \parbox{8.0cm}{Orientador: Heraldo Luis Silveira de Almeida} \\
   \vspace{2cm}
\begin{center}
Rio de Janeiro

Fevereiro de 2020
\end{center}


\pagebreak


\begin{center}
\large{\titulo{}}\\
   \vspace{1cm}
\large{Varlen Pavani Neto}\\
\end{center}
   \vspace{2cm}
PROJETO DE GRADUAÇÃO SUBMETIDO AO CORPO DOCENTE DO CURSO DE ENGENHARIA ELETRÔNICA E DE COMPUTAÇÃO DA ESCOLA POLITÉCNICA DA UNIVERSIDADE FEDERAL DO RIO DE JANEIRO COMO PARTE DOS REQUISITOS NECESSÁRIOS PARA A OBTENÇÃO DO GRAU DE ENGENHEIRO ELETRÔNICO E DE COMPUTAÇÃO   
   
   \vspace{1cm}
Autor:
      \vspace{0.5cm}
      \begin{flushright}
         \parbox{10cm}{
            \hrulefill

            \vspace{-.375cm}
            \centering{Varlen Pavani Neto}

            \vspace{0.1cm}
         }
      \end{flushright}
      
      
Orientador:
      \vspace{0.5cm}
      \begin{flushright}
         \parbox{10cm}{
            \hrulefill

            \vspace{-.375cm}
            \centering{Prof. Heraldo Luis Silveira de Almeida}

            \vspace{0.1cm}
         }
      \end{flushright}
      
Examinador:
      \vspace{0.5cm}
      \begin{flushright}
         \parbox{10cm}{
            \hrulefill

            \vspace{-.375cm}
            \centering{Prof xxxxx}

            \vspace{0.1cm}
         }
      \end{flushright}
      
Examinador:
      \vspace{0.5cm}
      \begin{flushright}
         \parbox{10cm}{
            \hrulefill

            \vspace{-.375cm}
            \centering{Prof xxxx}

            \vspace{0.1cm}
         }
      \end{flushright}
      
                        
      \vfill
      
      
\begin{center}
Rio de Janeiro

Julho de 2020
\end{center}



\include{Preambulo}

% Table of Contents
% ---------------------------------------------------------------
\tableofcontents
% ---------------------------------------------------------------
% Lista de figuras
% ---------------------------------------------------------------
%\cleardoublepage
%\addcontentsline{toc}{chapter}{Lista de Figuras}
\listoffigures
% ---------------------------------------------------------------
% Lista de Tabelas
% ---------------------------------------------------------------
%\cleardoublepage
%\addcontentsline{toc}{chapter}{Lista de Tabelas}
\listoftables

\mainmatter
\cleardoublepage


\paragraph{} O presente cap�tulo tem como objetivo apresentar brevemente o escopo do trabalho desenvolvido assim como sua motiva��o, introduzindo conceitos pertinentes � �rea de conhecimento.

\section{Tema}

\paragraph{} Este projeto tem como tema o estudo de t�cnicas para anonimiza��o e dessensibiliza��o em conjuntos de dados. Especificamente, ser�o avaliadas diferentes t�cnicas e suas implementa��es em software de c�digo aberto. 

\paragraph{} Este � um trabalho majoritariamente de Engenharia de Software, contemplando um ciclo de vida para planejamento, pesquisa e implementa��o da solu��o proposta.

\paragraph{} Tamb�m � poss�vel dizer que se trata de um trabalho da �rea de Engenharia de Dados, dada a natureza das entidades que ser�o manipuladas.  Tamb�m ser�o vistos conceitos da �rea de Seguran�a Digital, especificamente Direito Digital e Privacidade.

\paragraph{} Assim, este projeto implementa uma nova ferramenta de anonimiza��o e dessensibiliza��o de c�digo aberto englobando mais funcionalidades do que outras ferramentas existentes atualmente.


\section{Delimita��o}

\paragraph{} Este trabalho se limita a estudar t�cnicas de anonimiza��o e dessensibiliza��o, criadas at� o presente momento de sua concep��o, que permitam compatibilizar um banco de dados contendo informa��es sens�veis a uso como massa de dados para testes de software.

\section{Justificativa}

\paragraph{}Apresentar o porqu� do tema ser interessante de ser estudado. Cuidado, n�o � a motiva��o particular. Devem ser apresentadas raz�es para que algu�m deva se interessar no assunto, e n�o quais foram suas raz�es particulares que motivaram voc� a estud�-lo (tamanho do texto: livre).


\section{Objetivos}

\paragraph{}Informar qual � o objetivo geral do trabalho, isto �, aquilo que deve ser atendido e que corresponde ao indicador inequ�voco do sucesso do seu trabalho. Pode acontecer que venha a existir um conjunto de objetivos espec�ficos, que complementam o objetivo geral (tamanho do texto: livre, mas cuidado para n�o fazer uma literatura romanceada, afinal esta se��o trata dos objetivos).


\section{Metodologia}

\paragraph{}Como � a abordagem do assunto. Como foi feita a pesquisa, se vai houve valida��o, etc. Em resumo, voc� de explicar qual foi sua estrat�gia para atender ao objetivo do trabalho (tamanho do texto: livre).


\section{Descri��o}

\paragraph{}No cap�tulo 2 ser� .....

\paragraph{}O cap�tulo 3 apresenta ...

\paragraph{}Os .... s�o apresentados no cap�tulo 4. Nele ser� explicitado ...

\paragraph{}E assim vai at� chegar na conclus�o.

\chapter{Fundamentação Teórica}
\label{cap2}

\paragraph{} O presente capítulo propõe-se a introduzir o leitor ao entendimento acadêmico existente sobre o conceito de privacidade, especificamente no que contempla a área de Tecnologia da Informação com foco nos pontos de interesse ao presente trabalho,
principalmente com relação a anonimização e desanonimização de informações pessoais. 
Visando uma maior familiarização do leitor ao domínio deste trabalho, também serão introduzidos conceitos de armazenamento de dados.

\section{Privacidade}

\subsection{A Sensibilização sobre privacidade}

\paragraph{} De acordo com \cite{lgpd-evandro}, a Web 2.0 e o consequente crescimento da cultura participativa
aonde os próprios usuários consumidores são também geradores de conteúdo implicou numa
percepção de risco relacionado a privacidade cada vez maior.

\paragraph{} A ampla disponibilidade de dados privados na Internet trouxe o tema às pautas dos debates políticos 
mundiais, culminando na implementação de regulamentações para definir qual é o tratamento adequado
a estas informações, de modo a impedir o abuso no seu uso e garantir o direito a privacidade
de cada indivíduo.


\subsection{O entendimento multifacetado de privacidade}

\paragraph{}

\subsection{A General Data Protection Regulation europeia}

\paragraph{}

\subsection{A Lei Geral de Protecao de Dados brasileira}

\paragraph{}

\section{Bancos de Dados}

\subsection{Registros}

\subsection{ETL}

\subsection{Anonimização}

% \section{Figuras}

% \paragraph{}Figuras (organogramas, fluxogramas, esquemas, desenhos, fotografias, gr�ficos, mapas, plantas e outros) constituem unidade aut�noma e explicam, ou complementam visualmente o texto, portanto, devem ser inseridas o mais pr�ximo poss�vel do texto a que se referem. Sua identifica��o dever� aparecer na parte inferior precedida da palavra designativa (figura), seguida de seu n�mero de ordem de ocorr�ncia, do respectivo t�tulo e/ou legenda e da fonte, se necess�rio, tal como na Figura \ref{FigDel}.

% \begin{figure}
% \begin{center}
% \parbox[htb]{13.0cm}
%   {
%   \begin{center}
%   \includegraphics[scale=1.0]{logo_del.eps}
%   \caption[\small{Logotipo do DEL. Fonte: DEL/Poli/UFRJ \cite{Meyer97}.}]{\label{FigDel} \small{Logotipo do DEL. Fonte: DEL/Poli/UFRJ \cite{Meyer97}.}}
%   \end{center}
%   }
% \end{center}
% \end{figure}

% \section{Tabelas}
% \paragraph{}As tabelas s�o elementos demonstrativos de s�ntese que apresentam informa��es tratadas estatisticamente constituindo uma unidade aut�noma. Em sua apresenta��o deve ser observado: (1) o t�tulo dever� ser colocado na parte inferior, precedido da palavra Tabela e de seu n�mero de ordem; (2) as fontes e eventuais notas aparecem em seu rodap�, ap�s o fechamento, utilizando-se o tamanho 10; (3) Devem ser inseridas o mais pr�ximo poss�vel do trecho a que se referem, tal como a Tabela \ref{TabIntranet}.

% \begin{table}[h]
% 	\begin{center}
% 	  \caption{Casos de ataques aos computadores da Intranet. Fonte: DEL/Poli/UFRJ \cite{Meyer97}.}
% 		\begin{tabular}{|c|c|c|}\hline
% 		  \textbf{N�mero I}P & \textbf{Ataques} & \textbf{Ataques bem sucedidos} \\ \hline \vspace{-1.0mm}
% 		  192.168.0.120 & 54 & 1 \\ \hline \vspace{-1.0mm}
% 		  192.168.0.123 & 36 & 2 \\ \hline \vspace{-1.0mm}
% 		  192.168.0.129 & 25 & 4 \\ \hline \vspace{-1.0mm}
% 		  192.168.0.130 & 16 & 0 \\ \hline \vspace{-1.0mm}
% 		  192.168.0.141 & 29 & 3 \\ \hline \vspace{-1.0mm}
% 		  \textbf{Total} & \textbf{160} & \textbf{10} \\ \hline
% 		\end{tabular}
% 	\end{center}
% 	\label{TabIntranet}
% \end{table}


% \section{Numera��o de p�ginas}

% \paragraph{}O aluno deve observar atentamente a numera��o de p�ginas de seu projeto. A primeira parte deste modelo de projeto final, composta pela dedicat�ria, agradecimento, resumo, abstract, siglas, sum�rio, lista de figuras e lista de tabelas, � numerada seq�encialmente utilizando algarismos romanos min�sculos. As demais folhas, descritas na segunda parte deste modelo, s�o numeradas seq�encialmente utilizando algarismos ar�bicos.

% \paragraph{}Contudo, exclusivamente para a segunda parte do modelo de projeto, � permitida uma numera��o alternativa na qual o aluno poder� numerar as p�ginas por cap�tulo. Por exemplo, a primeira p�gina deste Cap�tulo 2 - Informa��es Adicionais, poderia ser escrita como 2.1. Al�m disto, a p�gina seguinte seria 2.2 e a presente p�gina poderia ser escrita como 2.3. A p�gina do Ap�ndice A - O que � um ap�ndice, poderia ser escrita como A.1, enquanto que a primeira p�gina do ap�ndice B seria B.1. Neste caso alternativo espec�fico, a Bibliografia na dever� conter numera��o.


\chapter{Metodologia}
\label{cap3}

\section{Descoberta e Análise de ferramentas de anonimização}

\paragraph{}
Para descoberta de ferramentas de anonimização neste trabalho, foram utilizados projetos de código-aberto disponíveis no GitHub. 

Especificamente, foram consideradas as métricas de stars e forks dos repositórios como medida de popularidade para seleção de 4 projetos relativos a anonimização de dados. Todos os projetos apresentavam novas modificações no código-fonte em Março de 2020, indicando que são mantidos pelas comunidade. Buscou-se inferir os casos de uso, funcionalidades, usabilidade e tecnologias utilizadas a partir da documentação dos projetos. Consideradou-se a primeira linha dos arquivos README.md dos repositórios como nome do projeto. Os repositórios do github estão em parênteses ao lado dos títulos de cada um dos projetos.

\subsection{Anonymizer}

O código-fonte deste projeto está disponível em github.com/DivanteLtd/anonymizer.

Anonymizer foi desenvolvida na linguagem de programação Ruby pela compania européia Divante e opera em bancos de dados MySQL. Segundo a documentação, sua funcionalidade mais importante é a formatação de dados. A ferramenta substitui os dados originais por dados gerados.

A instalação da ferramenta é feita a partir de uma cópia do repositório de código. 

O processo de anonimização é feito a partir de um arquivo de dump do banco de dados. Este arquivo pode estar na mesma máquina que executa o processo ou em uma máquina remota. O usuário deve criar um arquivo no formato JSON com os parâmetros do processo.

A ferramenta permite substituir os valores nas tabelas por valores fixos, valores em branco ou gerados, de acordo com a categoria. São suportadas as categorias firstname, lastname, login, email, telephone, company, street, postcode, city, full{\_}address, vat{\_}id, ip, quote, website, iban, json, uniq{\_}email, uniq{\_}login, regon e pesel. Também é possível truncar todos os dados de uma tabela.

Também é provida a capacidade de executar um comando SQL arbitrário antes ou depois do processo.

- Qual banco é compatível?

\subsection{Data::Anonymization}

O código-fonte deste projeto está disponível em github.com/sunitparekh/data-anonymization/.

TODO

\subsection{ARX - Open Source Data Anonymization Software}

O código-fonte deste projeto está disponível em github.com/arx-deidentifier/arx.

TODO

\subsection{Presidio}

O código-fonte deste projeto está disponível em github.com/microsoft/presidio.

TODO
\chapter{Implementação}
\label{cap4}

\section{Requisitos}

\section{Arquitetura Proposta}

\section{Desenvolvimento}
\chapter{Validação}
\label{cap5}

\section{Dados de Teste}

\paragraph{} Para validação do sistema implementado, utilizou-se uma versão para Postgres do banco de dados Northwind\cite{northwindpg}.

\paragraph{} O modelo deste banco de dados foi inicialmente criado pela Microsoft para fins educacionais e representa a estrutura de uma loja, incluindo pedidos, clientes e funcionários. Pode-se considerar que os dados de um cliente, como por exemplo o endereço de seus pedidos, são PII e requerem anonimização.

\paragraph{} O banco de dados foi implantado através de um container Docker.

\section{Resultados}
\chapter{Conclusão}
\label{cap6}

\paragraph{} O presente trabalho apresentou a construção de uma nova ferramenta de código aberto para anonimização de banco de dados através da geração de dados sintéticos.

\paragraph{} A ferramenta pode ser utilizada por instituições interessadas em compatilhar a estrutura de um banco de dados existente sem disponibilizar seu conteúdo potencialmente sensível, estando assim em cumprimento com as leis de proteção de dados e protegendo a privacidade dos indivíduos cujas informações estão nessa base de dados em questão.

\paragraph{} Dentro dos critérios propostos inicialmente, a ferramenta mostrou-se satisfatória, possibilitando gerar bancos de dados similares ao original preenchidos com dados similares.

\paragraph{} Durante os testes, foi observado que a acurácia do classificador de tipos semânticos é um dos principais fatores a serem ajustados para melhorar a coerência dos conjuntos de dados gerados.

\paragraph{} Neste sentido, dentre possiveis melhorias a serem avaliadas estão a alteração do funcionamento do classificador semântico para levar em consideração todas as amostras de dados simultaneamente ao invés de analisar cada tabela em isolamento e também uma implementação de classificação a partir do uso de modelos de linguagem massivos (LLMs).

\paragraph{} Além disso, futuros trabalhos podem incluir outras melhorias da classificação semântica dos dados, melhorias de usabilidade como a criação de uma interface de usuário e também a expansão do suporte a outras implementações de bancos de dados SQL além do Postgres.

% ---------------------------------------------------------------
% Bibliografia
% ---------------------------------------------------------------
\normalsize
\cleardoublepage
\addcontentsline{toc}{chapter}{Bibliografia}
\bibliographystyle{coppe}
\bibliography{biblio}


% Apendices 

% \appendix

% % Apendice A

% \chapter{O que é um apêndice}
% \label{ApendiceA}
% \paragraph{}Elemento que consiste em um texto ou documento elaborado pelo autor, com o intuito de complementar sua argumentao, sem prejuízo do trabalho. São identificados por letras maiúsculas consecutivas e pelos respectivos títulos.

% % Apendice B

% \chapter{Encadernação do Projeto de Graduação}
% \label{ApendiceB}
% \include{ApendiceB}

% % Apendice C

% \chapter{O que é um anexo}
% \label{ApendiceC}
% \include{ApendiceC}

\backmatter

\end{document}