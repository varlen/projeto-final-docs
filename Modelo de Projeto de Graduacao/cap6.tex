\chapter{Conclusão}
\label{cap6}

\paragraph{} O presente trabalho apresentou a construção de uma nova ferramenta de código aberto para anonimização de banco de dados através da geração de dados sintéticos.

\paragraph{} A ferramenta pode ser utilizada por instituições interessadas em compatilhar a estrutura de um banco de dados existente sem disponibilizar seu conteúdo potencialmente sensível, estando assim em cumprimento com as leis de proteção de dados e protegendo a privacidade dos indivíduos cujas informações estão nessa base de dados em questão.

\paragraph{} Dentro dos critérios propostos inicialmente, a ferramenta mostrou-se satisfatória, possibilitando gerar bancos de dados similares ao original preenchidos com dados similares.

\paragraph{} Durante os testes, foi observado que a performance do classificador de tipos semânticos é um dos principais fatores a serem ajustados para melhorar a coerência dos conjuntos de dados gerados.

\paragraph{} Neste sentido, uma possível melhoria a ser avaliada é a alteração do funcionamento do classificador semântico para levar em consideração todas as amostras de dados simultaneamente ao invés de analisar cada tabela em isolamento.

\paragraph{} Além disso, futuros trabalhos podem incluir outras melhorias da classificação semântica dos dados, melhorias de usabilidade como a criação de uma interface de usuário e também a expansão do suporte a outros dialetos de bancos de dados SQL.