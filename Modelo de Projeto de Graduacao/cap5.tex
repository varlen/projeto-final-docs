\chapter{Validação}
\label{cap5}

\section{Metodologia}

\paragraph{} A etapa inicial da avaliação foi a busca e descoberta de bancos de dados de exemplo disponíveis em repositórios públicos.

\paragraph{} Os bancos de dados selecionados representam informações estruturadas de organizações comerciais fictícias.

\paragraph{} Estes bancos foram selecionados por conterem informações que apesar de fictícias, são consideradas PII como por exemplo o endereço de um cliente ou funcionário, permitindo assim avaliar a capacidade de geração de um conjunto de dados anonimizado do programa.

\paragraph{} Um outro critério para escolha dos bancos de dados foi a presença de uma quantidade razoável de tabelas e relações entre dados.pelo

\paragraph{} Os seguinte bancos de dados foram selecionados:

\begin{enumerate}
    \item Northwind\cite{northwindpg} - Este banco de dados contém as informações de uma loja, incluindo clientes, funcionários e pedidos.
    \item Sakila\cite{sakila} - Este banco contém as informações de uma rede de locadoras de filmes, contendo tabelas como loja, cliente, pagamentos, etc.
\end{enumerate}

\paragraph{} Os banco de dados foram disponibilizados através de containers Docker.

\paragraph{} Também foram criados \textit{scripts} auxiliares contendo as variáveis de ambiente necessárias para execução do programa, contendo as configurações de caminho e conexão com os bancos de dados de origem e destino.

\section{Objetivos}

\paragraph{} O objetivo dos testes realizados foi validar os programas implementados em relação aos casos de uso inicialmente propostos.

\paragraph{} Os critérios considerados para avaliação da solução proposta foram:

\begin{enumerate}
    \item A possibilidade de usar os programas criados para que o responsável por um banco de dados sensível possa fornecer um banco de dados com a mesma estrutura e dados análogos à um desenvolvedor para o propósito de desenvolver sistemas, sem o risco de vazamento dos dados originais.
    \item A similaridade entre os bancos de dados com relação ao conteúdo semântico e sua utilidade para propósitos de desenvolvimento de software.
\end{enumerate}

\section{Resultados}

\paragraph{} A validação foi realizada a partir da execução do programa apresentado e a posterior inspeção do arquivo de metadados além da estrutura e conteúdo do banco de dados gerado.

\subsection{Northwind}

\paragraph{} Durante o teste com o banco de dados Northwind, o programa gerou toda estrutura de tabelas, colunas e relações existentes com sucesso.

\paragraph{} Com relação ao conteúdo do banco de dados gerado, a percepção de significado dos dados é fortemente dependente da precisão do componente de extração semântica.

\paragraph{} Como evidência do comportamento apurado no teste neste primeiro banco de dados, foi possível verificar que algumas colunas como \textit{country} e \textit{city} foram preenchidas com nomes de países e cidades, enquanto outras colunas como \textit{phone\_number} e \textit{address} foram preenchidas errôneamente com números de ponto flutuante e nomes de pessoas, respectivamente.

\subsection{Sakila}

\paragraph{} O teste com o segundo banco de dados evidenciou correções necessárias no código-fonte para processar corretamente colunas cujo tipo definido na tabela é \textit{decimal} além de trabalhar com colunas cujo tipo foi definido pelo usuário.

\paragraph{} A coluna \textit{rating} da tabela \textit{film} foi criada como uma coluna de tipo definido pelo usuário. Isso significa que o seu tipo foi criado na inicialização do banco de dados e não existe por padrão no motor do banco de dados.

\paragraph{} Como forma de suportar parcialemente tipos definidos pelo usuário, a aplicação foi modificada para tratar estes tipos da mesma maneira que o tipo nativo \textit{text}.

\paragraph{} Esta alteração permite trabalhar com tipos criados pelo usuário desde que sejam textuais e enumerações, como é o caso no banco de dados processado neste teste, mas não cobre toda a gama de possibilidades de criação de tipos, não sendo aplicável quando o tipo definido pelo usuário é uma agregação de tipos numéricos.