\chapter{Validação}
\label{cap5}

\section{Dados de Teste}


\paragraph{} Um trabalho de descoberta de bancos de dados de exemplo disponíveis em repositórios públicos foi realizado inicialmente.

\paragraph{} Os bancos de dados selecionados representam informações estruturadas de organizações comerciais fictícias.

\paragraph{} Estes bancos contém informações consideradas PII como por exemplo o endereço de um cliente ou funcionário.

\paragraph{} Específicamente os seguinte bancos de dados foram selecionados:

\begin{enumerate}
    \item Northwind\cite{northwindpg} - Este banco de dados contém as informações de uma loja, incluindo clientes, funcionários e pedidos.
    \item   \cite{sakila} - Este banco contém as informações de uma rede de locadoras de filmes, contendo tabelas como loja, cliente, pagamentos, etc.
    \item TODO
\end{enumerate}

\paragraph{} Os banco de dados foram disponibilizados através de containers Docker.

\section{Objetivos}

\paragraph{} O objetivo dos testes realizados foi validar os programas implementados em relação aos casos de uso inicialmente propostos.

\paragraph{} Os seguinte critérios foram considerados para avaliação da solução proposta:

\begin{enumerate}
    \item A possibilidade de usar os programas criados para que o responsável por um banco de dados sensível possa fornecer um banco de dados com a mesma estrutura e dados análogos à um desenvolvedor para o propósito de desenvolver sistemas, sem o risco de vazamento dos dados originais.
    \item A similaridade entre os bancos de dados com relação ao conteúdo semântico e sua utilidade para propósitos de desenvolvimento de software.
\end{enumerate}

\section{Resultados}

\paragraph{} TODO